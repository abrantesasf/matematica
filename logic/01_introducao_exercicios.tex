%%%%%%%%%%%%%%%%%%%%%%%%%%%%%%%%%%%%%%%%%%%%%%%%%%%%%%%%%%%%%%%%%%
% Arquivo LaTeX geral para a classe Article
%
% Abrantes Araújo Silva Filho
% abrantesasf@gmail.com
% 2018-02-25


%%%%%%%%%%%%%%%%%%%%%%%%%%%%%%%%%%%%%%%%%%%%%%%%%%%%%%%%%%%%%%%%%%
%%% Configura tipo de documento e load de packages:
\RequirePackage{ifpdf}
\ifpdf
  \documentclass[pdftex,a4paper,12pt,brazil]{article} % Se tem draft é rascunho
  %\usepackage{ae}
  \usepackage[pdftex]{geometry}
  \geometry{a4paper,left=2cm,right=2cm,top=2cm,bottom=2cm}
  \usepackage[pdftex]{graphicx}
  \usepackage{setspace}
  \usepackage[T1]{fontenc}
  \usepackage[utf8]{inputenc}
  \usepackage[brazil]{babel}
  \usepackage[brazil]{varioref}
  \usepackage[pdftex,pdfpagemode=UseOutlines,bookmarks=true,%
   bookmarksopen=true,bookmarksopenlevel=5,bookmarksnumbered=true,%
   pdfstartview=FitH,hyperfootnotes=true]{hyperref}
   \hypersetup{pdfinfo={
   Author={Abrantes Ara\'{u}jo Silva Filho},
   Title={Respostas de exerc\'{i}cios selecionados de lógica matem\'{a}tica},
   Creator={pdfLaTeX},
   Producer={pdfTeX},
   CreationDate={},
   ModDate={},
   Subject={Estudo sobre l\'{o}gica matem\'{a}tica},
   Keywords={l\'{o}gica, logic, respostas, solutions},
   }}
  %\usepackage{thumbpdf}
  \hypersetup{colorlinks,%
    debug=false,%
    linkcolor=blue,%
    citecolor=blue,%
    urlcolor=blue}
  \usepackage{cleveref}
  \mathchardef\period=\mathcode`.
\else
  \documentclass[a4paper,12pt]{article}
  \usepackage[utf8]{inputenc}
  \usepackage[T1]{fontenc}
  \usepackage[brazil]{babel}
  \usepackage[dvips]{geometry}
  \usepackage[dvips]{graphicx}
  \geometry{a4paper,left=2cm,right=2cm,top=2cm,bottom=2cm}
  \usepackage{setspace}
  \usepackage{varioref}
  \usepackage{hyperref}
  \usepackage{cleveref}
\fi


%%%%%%%%%%%%%%%%%%%%%%%%%%%%%%%%%%%%%%%%%%%%%%%
%%% Configura lingua portuguesa:
%\usepackage[brazil]{babel}
%\usepackage[utf8]{inputenc}
%\usepackage[T1]{fontenc}


%%%%%%%%%%%%%%%%%%%%%%%%%%%%%%%%%%%%%%%%%%%%%%%
%%% Altera fonte padrão
% phv=Helvetica ptm=Times ppl=Palatino pbk=bookman
% pag=AdobeAvantGarde pnc=Adobe NewCenturySchoolbook
\renewcommand{\familydefault}{ppl}


%%%%%%%%%%%%%%%%%%%%%%%%%%%%%%%%%%%%%%%%%%%%%%%
%%% Configura símbolos e bibliotecas matemáticas:
\usepackage{amsmath}
\usepackage{amssymb}
\usepackage{latexsym}
\usepackage{array}
\usepackage[ruled]{algorithm}
%\usepackage{physics}
\usepackage{syllogism}
\usepackage{mathpartir}
\usepackage{siunitx}
\sisetup{group-separator = {.}}
\sisetup{group-digits = {false}}
\sisetup{output-decimal-marker = {,}}

\usepackage{listings}
\lstset{literate=
  {á}{{\'a}}1 {é}{{\'e}}1 {í}{{\'i}}1 {ó}{{\'o}}1 {ú}{{\'u}}1
  {Á}{{\'A}}1 {É}{{\'E}}1 {Í}{{\'I}}1 {Ó}{{\'O}}1 {Ú}{{\'U}}1
  {à}{{\`a}}1 {è}{{\`e}}1 {ì}{{\`i}}1 {ò}{{\`o}}1 {ù}{{\`u}}1
  {À}{{\`A}}1 {È}{{\'E}}1 {Ì}{{\`I}}1 {Ò}{{\`O}}1 {Ù}{{\`U}}1
  {ä}{{\"a}}1 {ë}{{\"e}}1 {ï}{{\"i}}1 {ö}{{\"o}}1 {ü}{{\"u}}1
  {Ä}{{\"A}}1 {Ë}{{\"E}}1 {Ï}{{\"I}}1 {Ö}{{\"O}}1 {Ü}{{\"U}}1
  {â}{{\^a}}1 {ê}{{\^e}}1 {î}{{\^i}}1 {ô}{{\^o}}1 {û}{{\^u}}1
  {Â}{{\^A}}1 {Ê}{{\^E}}1 {Î}{{\^I}}1 {Ô}{{\^O}}1 {Û}{{\^U}}1
  {œ}{{\oe}}1 {Œ}{{\OE}}1 {æ}{{\ae}}1 {Æ}{{\AE}}1 {ß}{{\ss}}1
  {ű}{{\H{u}}}1 {Ű}{{\H{U}}}1 {ő}{{\H{o}}}1 {Ő}{{\H{O}}}1
  {ç}{{\c c}}1 {Ç}{{\c C}}1 {ø}{{\o}}1 {å}{{\r a}}1 {Å}{{\r A}}1
  {€}{{\euro}}1 {£}{{\pounds}}1 {«}{{\guillemotleft}}1
  {»}{{\guillemotright}}1 {ñ}{{\~n}}1 {Ñ}{{\~N}}1 {¿}{{?`}}1
}


%%%%%%%%%%%%%%%%%%%%%%%%%%%%%%%%%%%%%%%%%%%%%%%
%%% Configura fontes e outros símbolos
\usepackage{wasysym}
\usepackage{pifont}
\usepackage{marvosym}


%%%%%%%%%%%%%%%%%%%%%%%%%%%%%%%%%%%%%%%%%%%%%%%
%%% Ativa pacote ifthen, necessário para alguns comandos
\usepackage{ifthen}


%%%%%%%%%%%%%%%%%%%%%%%%%%%%%%%%%%%%%%%%%%%%%%%
%%% Ativa suporte a cores:
\usepackage{color}
\usepackage[dvipsnames]{xcolor}
\usepackage{xparse}


%%%%%%%%%%%%%%%%%%%%%%%%%%%%%%%%%%%%%%%%%%%%%%%
%%% Ativa figuras e tabelas
\usepackage{float}
\usepackage{wrapfig}


%%%%%%%%%%%%%%%%%%%%%%%%%%%%%%%%%%%%%%%%%%%%%%%
%%% Ativa suporte ao TikZ Code
\usepackage{tikz}
\usetikzlibrary{positioning,shapes,shadows}


%%%%%%%%%%%%%%%%%%%%%%%%%%%%%%%%%%%%%%%%%%%%%%%
%%% Ativa pacote para tabelas longas e em landscape
\usepackage{array,longtable}
\usepackage{lscape}
\usepackage{array}
\usepackage{colortbl}
\newcolumntype{M}[1]{>{\centering\arraybackslash}m{#1}}
%\newcolumntype{ML}[1]{>{$}l<{$}}
%\newcolumntype{MR}[1]{>{R}r<{R}}
\newcolumntype{L}[1]{>{\arraybackslash}m{#1}}
\newcolumntype{N}{@{}m{0pt}@{}}


%%%%%%%%%%%%%%%%%%%%%%%%%%%%%%%%%%%%%%%%%%%%%%%
%%% Ativa pacote para URLs, e-mails e pathmanes:
\usepackage{url}


%%%%%%%%%%%%%%%%%%%%%%%%%%%%%%%%%%%%%%%%%%%%%%%
%%% Commando para ``italizar´´ palavras em inglês (e outras línguas!)
\newcommand{\ingles}[1]{\textit{#1}}


%%%%%%%%%%%%%%%%%%%%%%%%%%%%%%%%%%%%%%%%%%%%%%%
%%% Commando para colocar o espaço correto entre um número e sua unidade
\newcommand{\unidade}[2]{\ensuremath{#1\,\mathrm{#2}}}
\newcommand{\unidado}[2]{{#1}\,{#2}}


%%%%%%%%%%%%%%%%%%%%%%%%%%%%%%%%%%%%%%%%%%%%%%%%%%%%%%%%%%%%
%% produz ordinal masculino ou feminino dependendo do segundo
%% argumento.  Por exemplo:
%% \ordinal{1}{a} Semana
%% \ordinal{1}{o} Encontro
\newcommand{\ordinal}[2]{%
#1%
\ifthenelse{\equal{a}{#2}}%
{\textordfeminine}%
{\textordmasculine}}


%%%%%%%%%%%%%%%%%%%%%%%%%%%%%%%%%%%%%%%%%%%%%%%
%%% Ativa suporte a sublinhado:
% A opção normalem indica que ênfase será dada por itálico
% e não por sublinhado.
\usepackage[normalem]{ulem}
% O código a seguir define mais comandos para o pacote ulem, e foi retirado
% do "LaTeX demo: exemplos com LaTeXe", de Klauss Steding Jessen:
\def\dotuline{\bgroup
  \ifdim\ULdepth=\maxdimen  % Set depth based on font, if not set already
  \settodepth\ULdepth{(j}\advance\ULdepth.4pt\fi
  \markoverwith{\begingroup
  \advance\ULdepth0.08ex
  \lower\ULdepth\hbox{\kern.15em .\kern.1em}%
  \endgroup}\ULon}
\def\dashuline{\bgroup
  \ifdim\ULdepth=\maxdimen  % Set depth based on font, if not set already
  \settodepth\ULdepth{(j}\advance\ULdepth.4pt\fi
  \markoverwith{\kern.15em
  \vtop{\kern\ULdepth \hrule width .3em}%
  \kern.15em}\ULon}


%%%%%%%%%%%%%%%%%%%%%%%%%%%%%%%%%%%%%%%%%%%%%%%
%%% Ativa pacote para indentação da primeira linha de parágrafos
%\usepackage{indentfirst}


%%%%%%%%%%%%%%%%%%%%%%%%%%%%%%%%%%%%%%%%%%%%%%%
%%% Ativa pacote enumerate, extensão ao environment enumerate:
\usepackage{enumerate}


%%%%%%%%%%%%%%%%%%%%%%%%%%%%%%%%%%%%%%%%%%%%%%%
%%% environment ``Description'', similar ao environment
%%% ``description'', mas com maior controle sobre a tabulação das
%%% entradas e de suas descrições.
%%% Adaptado de um exemplo do LaTeX Companion, pg. 64.
\newlength{\myentrylen}
\newenvironment{Description}[1]%
{\list{}
  {\settowidth{\labelwidth}{\textbf{#1}}%
    \leftmargin\labelwidth\advance\leftmargin\labelsep%
    \renewcommand{\makelabel}[1]{%
      \settowidth{\myentrylen}{\textbf{##1}}%
      \ifthenelse{\lengthtest{\myentrylen > \labelwidth}}%
      {\parbox[b]{\labelwidth}%
        {\makebox[0pt][l]{\textbf{##1}}\\\mbox{}}}
      {\textbf{##1}}%
      \hfill\relax%
      }
}}
{\endlist}


%%%%%%%%%%%%%%%%%%%%%%%%%%%%%%%%%%%%%%%%%%%%%%%
%%% Comando para epígrafe em capítulos/seções (não confundir com a epígrafe geral
% da tese, definida em página isolada nos elementos pré-textuais.
\newcommand{\epigrafe}[2]{
   \vspace{-6ex}%
     {\footnotesize%
     \begin{flushright}%
     \begin{minipage}{.6\textwidth}%
     #1
     \end{minipage}\\
     \textit{#2}%
     \end{flushright}}%
   \vspace{-3ex}}


%%%%%%%%%%%%%%%%%%%%%%%%%%%%%%%%%%%%%%%%%%%%%%%
%%% Ativa pacote para controle de cabeçalhos e rodapés e configura;
% Ativa pacote:
\usepackage{fancyhdr}
% Configura estilo padrão das páginas
\pagestyle{headings}


%%%%%%%%%%%%%%%%%%%%%%%%%%%%%%%%%%%%%%%%%%%%%%%
%%% Ativa pacote para formatar os captions:
\usepackage[normal,bf]{caption}
\captionsetup[table]{font=small,skip=0pt}
\captionsetup[figure]{skip=0pt}


%%%%%%%%%%%%%%%%%%%%%%%%%%%%%%%%%%%%%%%%%%%%%%%
%%% Ativa o MakeIndex para fazer índices remissivos:
\usepackage{makeidx}
%\makeindex


%%%%%%%%%%%%%%%%%%%%%%%%%%%%%%%%%%%%%%%%%%%%%%%
%%% Ativa pacote para fazer glossário, conforme
% "LaTeX demo: exemplos com LaTeXe", de Klauss Steding Jessen.
\usepackage{makeglo}
%\makeglossary


%%%%%%%%%%%%%%%%%%%%%%%%%%%%%%%%%%%%%%%%%%%%%%%
%%% Ativa o pacote havard de referências bibliográficas e
% define um novo comando para colocar as citações em slanted:
% As opções do pacote determinam como as citações aparecem no texto.
% As seguinte opções existem:
%      default: lista a primeira completa e as subsequentes abreviadas;
%	  full: lista todas as citações completas;
%         abbr: lista todas as citações abreviadas.
% A qualquer momento o modo de citaçõa pode ser alterado com o uso do
% comando: \citationmode{}, cujo argumento é uma das opções da lista anterior.
\usepackage[default]{harvard}
% Cria comando para colocar as citações em slanted:
\newcommand{\refbib}[1]{\textsl{#1}}
% O seguinte comando configura como as referências bibliográficas serão
% formatadas. O estilo agsm é o padrão do pacote havard. Ver manual de instrução
% do pacote para maiores informações.
\bibliographystyle{agsm}
% Configura como as citações das referências apareceram no texto. O estilo agsm
% é o padrão do pacote havard. Ver manual de instrução do pacote para maiores
% informações.
\citationstyle{agsm}


%%%%%%%%%%%%%%%%%%%%%%%%%%%%%%%%%%%%%%%%%%%%%%%
%%% Comandos específicos para este documento


%%%%%%%%%%%%%%%%%%%%%%%%%%%%%%%%%%%%%%%%%%%%%%%
%%% Determina forma de hifenização de palavras quando a hifenização
%%% padrão não estiver correta
%\hyphenation{ne-nhu-ma}
\babelhyphenation[brazil]{ne-nhu-ma Git-Hub}


%%%%%%%%%%%%%%%%%%%%%%%%%%%%%%%%%%%%%%%%%%%%%%%%%%%%%%%%%%%%%%%%%%%%%%%%%%%%%%%%%%%%%%%%%%%%%%
%%%%%%%%%%%%%%%%%%%%%%%%%%%%%%%%%%%%%%%%%%%%%%%%%%%%%%%%%%%%%%%%%%%%%%%%%%%%%%%%%%%%%%%%%%%%%%
%%%%%%%%%%%%%%%%%%%%%%%%%%%%%%%%%%%%%%%%%%%%%%%%%%%%%%%%%%%%%%%%%%%%%%%%%%%%%%%%%%%%%%%%%%%%%%
%%%%%%%%%%%%%%%%%%%%%%%%%%%%%%%%%%%%%%%%%%%%%%%%%%%%%%%%%%%%%%%%%%%%%%%%%%%%%%%%%%%%%%%%%%%%%%
%%%%%%%%%%%%%%%%%%%%%%%%%%%%%%%% COMEÇA DOCUMENTO %%%%%%%%%%%%%%%%%%%%%%%%%%%%%%%%%%%%%%%%%%%%
%%%%%%%%%%%%%%%%%%%%%%%%%%%%%%%%%%%%%%%%%%%%%%%%%%%%%%%%%%%%%%%%%%%%%%%%%%%%%%%%%%%%%%%%%%%%%%
%%%%%%%%%%%%%%%%%%%%%%%%%%%%%%%%%%%%%%%%%%%%%%%%%%%%%%%%%%%%%%%%%%%%%%%%%%%%%%%%%%%%%%%%%%%%%%
%%%%%%%%%%%%%%%%%%%%%%%%%%%%%%%%%%%%%%%%%%%%%%%%%%%%%%%%%%%%%%%%%%%%%%%%%%%%%%%%%%%%%%%%%%%%%%
%%%%%%%%%%%%%%%%%%%%%%%%%%%%%%%%%%%%%%%%%%%%%%%%%%%%%%%%%%%%%%%%%%%%%%%%%%%%%%%%%%%%%%%%%%%%%%
\begin{document}
\title{Exercícios de introdução à \emph{Lógica Simbólica Dedutiva}\\
  (várias fontes)\\
  \ \\
--- respostas de exercícios selecionados ---}
\author{Abrantes Araújo Silva Filho}
\date{2018-03}
\maketitle
\tableofcontents
%\newpage


%%%%%%%%%%%%%%%%%%%%%%%%%%%%%%%%%%%%%%%%%%%%%%%%%%%%%%%%%%%%%%%%%%%%%%%%%%%%%%%%%%%%%%%%%%%%%%
%%%%%%%%%%%%%%%%%%%%%%%%%%%%%%%%%%%%%%%%%%%%%%%%%%%%%%%%%%%%%%%%%%%%%%%%%%%%%%%%%%%%%%%%%%%%%%
%%%%%%%%%%%%%%%%%%%%%%%%%%%%%%%%%%%%%%%%%%%%%%%%%%%%%%%%%%%%%%%%%%%%%%%%%%%%%%%%%%%%%%%%%%%%%%
%%%%%%%%%%%%%%%%%%%%%%%%%%%%%%%%%%%%%%%%%%%%%%%%%%%%%%%%%%%%%%%%%%%%%%%%%%%%%%%%%%%%%%%%%%%%%%
%%%%%%%%%%%%%%%%%%%%%%%%%%%%%%%%%%%%%%%%%%%%%%%%%%%%%%%%%%%%%%%%%%%%%%%%%%%%%%%%%%%%%%%%%%%%%%
\section{O que é este documento?} 
\label{o_que_e}
%\thispagestyle{plain}

Este documento contém as minhas respostas aos exercícios e problemas introdutórios de 
\emph{Lógica Simbólica Dedutiva}, presentes em três fontes:

\begin{itemize}
\item \emph{The Logic Book}, de Merrie Bergmann, James Moor e Jack Nelson (\unidado{6}{a}
  edição);
\item \emph{Introdução à Lógica Matemática}, de Rogério Miguel Coelho (\unidado{1}{a}
  edição);
\item \emph{Logic I (MIT 24.241)}, curso de lógica do MIT, disponível em
  \url{https://ocw.mit.edu/courses/linguistics-and-philosophy/24-241-logic-i-fall-2009/}
\end{itemize}

ATENÇÃO: não garanto que tudo aqui está correto, pelo contrário, algumas respostas expressam
minha visão particular e podem estar em desacordo com
a ``resposta padrão'' dos autores do livro ou do professor da disciplina. Também
não garanto que todos os exercícios e problemas do capítulo ou livro estarão resolvidos aqui.
De qualquer modo, caso pretenda
utilizar este documento como base para seu próprio estudo, tenha em mente o seguinte:

\begin{quote}
  \emph{Este documento é fornecido ``no estado em que se encontra'', sem garantias de qualquer
    natureza, expressas ou implícitas. Em nenhuma hipótese o autor poderá ser responsabilizado
    por qualquer problema, dano, prejuízo material ou imaterial decorrente do uso deste
    conteúdo.}
\end{quote}

Este documento (em formato PDF), o original em \LaTeX\ e outros materiais
adicionais (se necessário) estão disponíveis no seguinte
repositório GitHub: \url{https://github.com/abrantesasf/matematica} (procure pelo
diretório ``logic'').


%%%%%%%%%%%%%%%%%%%%%%%%%%%%%%%%%%%%%%%%%%%%%%%%%%%%%%%%%%%%%%%%%%%%%%%%%%%%%%%%%%%%%%%%%%%%%%
%%%%%%%%%%%%%%%%%%%%%%%%%%%%%%%%%%%%%%%%%%%%%%%%%%%%%%%%%%%%%%%%%%%%%%%%%%%%%%%%%%%%%%%%%%%%%%
%%%%%%%%%%%%%%%%%%%%%%%%%%%%%%%%%%%%%%%%%%%%%%%%%%%%%%%%%%%%%%%%%%%%%%%%%%%%%%%%%%%%%%%%%%%%%%
%%%%%%%%%%%%%%%%%%%%%%%%%%%%%%%%%%%%%%%%%%%%%%%%%%%%%%%%%%%%%%%%%%%%%%%%%%%%%%%%%%%%%%%%%%%%%%
%%%%%%%%%%%%%%%%%%%%%%%%%%%%%%%%%%%%%%%%%%%%%%%%%%%%%%%%%%%%%%%%%%%%%%%%%%%%%%%%%%%%%%%%%%%%%%
\section{Exercícios do \emph{The Logic Book}, capítulo 1}
\label{tlb-1}
%\thispagestyle{plain}


%%%%%%%%%%%%%%%%%%%%%%%%%%%%%%%%%%%%%%%%%%%%%%%%%%%%%%%%%%%%%%%%%%%%%%%%%%%%%%%%%%%%%%%%%%%%%%
%%%%%%%%%%%%%%%%%%%%%%%%%%%%%%%%%%%%%%%%%%%%%%%%%%%%%%%%%%%%%%%%%%%%%%%%%%%%%%%%%%%%%%%%%%%%%%
\subsection{Seção 1.2E}
\label{tlb-1-12e}

\paragraph{1.a)} A sentença é uma proposição e tem um valor-verdade (no caso, falso).

\paragraph{1.c)} A sentença é uma frase imperativa, não tem um valor-verdade, logo não é
uma proposição e está fora do escopo da lógica dedutiva.

\paragraph{1.e)} A sentença é uma proposição e tem um valor-verdade (no caso, verdadeiro).

\paragraph{1.g)} A sentença é uma proposição e tem um valor-verdade (no caso, falso).

\paragraph{2.a)} É um argumento. Forma padrão:

\begin{quote}
  Quando Mike, Sharon, Sandy e Vicky estão todos fora do escritório de Tacoma,
  nenhuma decisão importante é tomada.

  Mike está esquiando.

  Sharon está em Spokane.
  
  \vspace{-0.5cm}
  \noindent\hrulefill
  
  Assim, nenhuma decisão será tomada hoje.
\end{quote}

\paragraph{2.c)} Não é um argumento, são 3 premissas sem nenhuma conclusão.

\paragraph{2.e)} É um argumento. Forma padrão:

\begin{quote}
  Todos os parafusos que temos estão no gaveteiro.

  A primeira gaveta contém parafusos galvanizados.

  A segunda gaveta contém parafusos comuns.

  A terceira gaveta contém parafusos para madeira.

  A quarta e quinta gavetas contém parafusos de latão.

  Para reparar o piso, precisamos de parafusos de aço.

  \vspace{-0.5cm}
  \noindent\hrulefill
  
  Portanto, não será possível reparar o deck.
\end{quote}

\paragraph{2.g)} É uma sentença interrogativa, portanto não é um argumento.

\paragraph{2.i)} É um argumento. Forma padrão:

\begin{quote}
  Se Sarah fez a fiação, esta está correta.

  Se Marcie fez o encanamento, este está correto.

  Nem a fiação nem o encanamento estão corretos.

  \vspace{-0.5cm}
  \noindent\hrulefill
  
  Portanto, Sarah não fez a fiação e Marcie não fez e encanamento.
\end{quote}

\paragraph{2.k)} É um argumento. Forma padrão:

\begin{quote}
  Ser curado de câncer requer ter câncer.

  Ser curado de câncer é bom.
  
  Tudo que é requerido por algo que é bom, é bom em si mesmo.

  \vspace{-0.5cm}
  \noindent\hrulefill
  
  Portanto, ter câncer é bom.
\end{quote}


%%%%%%%%%%%%%%%%%%%%%%%%%%%%%%%%%%%%%%%%%%%%%%%%%%%%%%%%%%%%%%%%%%%%%%%%%%%%%%%%%%%%%%%%%%%%%%
%%%%%%%%%%%%%%%%%%%%%%%%%%%%%%%%%%%%%%%%%%%%%%%%%%%%%%%%%%%%%%%%%%%%%%%%%%%%%%%%%%%%%%%%%%%%%%
\subsection{Seção 1.3E}
\label{tlb-1-13e}

\paragraph{1.a)} Verdadeiro. Essa conclusão é logicamente verdadeira independentemente
das premissas, assim o argumento é sempre válido. Toda vez que uma conclusão for logicamente
verdadeira, o argumento é sempre válido pois é impossível, quaisquer que sejam as premissas,
partir da verdade para a falsidade (pois o argumento é logicamente válido).

\paragraph{1.b)} Verdadeiro. As premissas formam um conjunto logicamente inconsistente,
ou seja, é impossível que as premissas sejam todas verdadeiras. Assim, é impossível partir
da verdade para a falsidade (já que nem todas as premissas são verdadeiras). Toda vez que
as premissas formam um conjunto logicamente inconsistente, o argumento é válido.

\paragraph{1.c)} Falso. A conclusão é logicamente inválida e nesse caso podem existir
argumentos onde nem todas as premissas são verdadeiras e a conclusão é falsa, ou seja,
existem argumentos logicamente válidos com conclusões falsas. Assim, é
impossível partir da verdade (já que uma ou mais premissas não são verdadeiras) para a
falsidade.

\paragraph{2.a)} Não, pois uma única premissa verdadeira não é suficiente para que o
argumento seja válido. Um argumento é válido se, e apenas se, for impossível que todas as
premissas sejam verdadeiras e a conclusão falsa. Assim, para ser um argumento válido, é necessário
que todas as premissas sejam verdadeiras, e não apenas uma delas.

\paragraph{2.c, parte 1:)} Se um argumento tem uma conclusão logicamente verdadeira,
o argumento é válido independentemente de suas premissas pois nesse caso é impossível
que premissas verdadeiras levem à conclusão falsa (pois a conclusão já é logicamente
verdadeira). Assim, esse tipo de argumento é válido.

\paragraph{2.c, parte 2:)} Alguns argumentos válidos são robustos e outros não. Um argumento
é robusto se, e apenas se, ele for válido e suas premissas corretas (ou seja, correspondem
realmente à verdade concreta). Outros argumentos podem ser válidos mas não robustos, ou seja,
são logicamente válidos mas concretamente não correspondem à verdade (são incorretos).


%%%%%%%%%%%%%%%%%%%%%%%%%%%%%%%%%%%%%%%%%%%%%%%%%%%%%%%%%%%%%%%%%%%%%%%%%%%%%%%%%%%%%%%%%%%%%%
%%%%%%%%%%%%%%%%%%%%%%%%%%%%%%%%%%%%%%%%%%%%%%%%%%%%%%%%%%%%%%%%%%%%%%%%%%%%%%%%%%%%%%%%%%%%%%
%%%%%%%%%%%%%%%%%%%%%%%%%%%%%%%%%%%%%%%%%%%%%%%%%%%%%%%%%%%%%%%%%%%%%%%%%%%%%%%%%%%%%%%%%%%%%%
%%%%%%%%%%%%%%%%%%%%%%%%%%%%%%%%%%%%%%%%%%%%%%%%%%%%%%%%%%%%%%%%%%%%%%%%%%%%%%%%%%%%%%%%%%%%%%
%%%%%%%%%%%%%%%%%%%%%%%%%%%%%%%%%%%%%%%%%%%%%%%%%%%%%%%%%%%%%%%%%%%%%%%%%%%%%%%%%%%%%%%%%%%%%%
\section{Exercícios do \emph{Introdução à Lógica Matemática}, capítulo 1}
\label{ilm-1}
%\thispagestyle{plain}


%%%%%%%%%%%%%%%%%%%%%%%%%%%%%%%%%%%%%%%%%%%%%%%%%%%%%%%%%%%%%%%%%%%%%%%%%%%%%%%%%%%%%%%%%%%%%%
%%%%%%%%%%%%%%%%%%%%%%%%%%%%%%%%%%%%%%%%%%%%%%%%%%%%%%%%%%%%%%%%%%%%%%%%%%%%%%%%%%%%%%%%%%%%%%
\subsection{Seção 1.6}
\label{ilm-1-16}

\paragraph{1.a)} É válida, pois temos como deduzir seu valor lógico.

\paragraph{1.b)} Não é uma proposição, é uma interrogação.

\paragraph{1.c)} Não é um proposição, pois não sabemos dizer com certeza qual é o servidor.

\paragraph{1.d)} É válida.

\paragraph{1.e)} É válida, mas não sabemos ainda seu valor lógico (é uma proposição indeterminada).

\paragraph{2.a)} Princípio da não contradição.

\paragraph{2.b)} Principio do terceiro excluído.

\paragraph{2.c)} Princípio da identidade.

\paragraph{3.a)} Atende ao princípio da identidade.

\paragraph{3.b)} Atende ao princípio do terceiro excluído.

\paragraph{3.c)} Atende ao princípio do terceiro excluído.

\paragraph{3.d)} Não atende aos postulados da lógica clássica (princípio
da não contradição).

\paragraph{3.e)} Atende ao princípio da identidade.

\paragraph{4.a)} Atende ao princípio da identidade.

\paragraph{4.b)} Não atende aos postulados da lógica clássica pois não é possível
saber se João estava com a luneta, ou se eu usei a luneta para vê-lo.

\paragraph{4.c)} Atende ao princípio da não contradição (não pode ser mortal
e imortal ao mesmo tempo).

\paragraph{4.d)} Atende ao princípio da não contradição (não pode ser palpável
e intocável ao mesmo tempo).

\paragraph{4.e)} Atende ao princípio da não contradição (os alunos não podem
saber e não saber ao mesmo tempo).

\paragraph{5.a)} É paradoxo.

\paragraph{5.b)} É paradoxo.

\paragraph{5.c)} Não é paradoxo, pois não é aparentemente verdadeira.

\paragraph{5.d)} É paradoxo.

\paragraph{5.e)} É paradoxo.

\paragraph{6.i)} Princípio do terceiro excluído.

\paragraph{6.ii)} Teorema.

\paragraph{6.iii)} Axioma.

\paragraph{6.iv)} Princípio da não contradição.

\paragraph{6.v)} Princípio da não contradição.

\paragraph{6.vi)} Princípio da identidade.

\paragraph{7.)} Sim, pois a frase indica que o barbeiro faz a barba
dos homens que não barbeiam a si próprios. Se isso for verdade, ele não
poderia fazer a própria barba.



%%%%%%%%%%%%%%%%%%%%%%%%%%%%%%%%%%%%%%%%%%%%%%%%%%%%%%%%%%%%%%%%%%%%%%%%%%%%%%%%%%%%%%%%%%%%%%
%%%%%%%%%%%%%%%%%%%%%%%%%%%%%%%%%%%%%%%%%%%%%%%%%%%%%%%%%%%%%%%%%%%%%%%%%%%%%%%%%%%%%%%%%%%%%%
%%%%%%%%%%%%%%%%%%%%%%%%%%%%%%%%%%%%%%%%%%%%%%%%%%%%%%%%%%%%%%%%%%%%%%%%%%%%%%%%%%%%%%%%%%%%%%
%%%%%%%%%%%%%%%%%%%%%%%%%%%%%%%%%%%%%%%%%%%%%%%%%%%%%%%%%%%%%%%%%%%%%%%%%%%%%%%%%%%%%%%%%%%%%%
%%%%%%%%%%%%%%%%%%%%%%%%%%%%%%%%%%%%%%%%%%%%%%%%%%%%%%%%%%%%%%%%%%%%%%%%%%%%%%%%%%%%%%%%%%%%%%
%\newpage
%\section{Problemas}
%\label{problemas}
%\thispagestyle{plain}


%%%%%%%%%%%%%%%%%%%%%%%%%%%%%%%%%%%%%%%%%%%%%%%%%%%%%%%%%%%%%%%%%%%%%%%%%%%%%%%%%%%%%%%%%%%%%%
%\subsection{Problema 2.1}
%\label{problema_2_1}


%%%%%%%%%%%%%%%%%%%%%%%%%%%%%%%%%%%%%%%%%%%%%%%
%%% Produz as referências bibliográficas
%% Configura título das referências bibliográficas:
%\newpage
%\renewcommand{\refname}{Referências bibliográficas}
%% Ativa arquivo com as referências bibliográficas:
%\bibliography{evandro}
%% Adiciona entrada na toc:
%\addcontentsline{toc}{section}{Referências bibliográficas}
%% Estilo da página
%\thispagestyle{headings}
%\index{referencias bibliograficas@Referências bibliográficas}


% Termina o documento
\end{document}             
